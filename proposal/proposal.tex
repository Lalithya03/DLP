\documentclass[notitlepage, 11pt]{article}

\title{COMP 5970/6970\\
Deep Learning Project Proposal}
\author{Tripp Isbell, Lalithya Kuntamukkala, James Lee}
\date{}

\usepackage{geometry}
\geometry{margin=1in}

\usepackage{titling}
\setlength{\droptitle}{-5em}

\pagenumbering{gobble}

\usepackage{cite}

\bibliographystyle{IEEEtran}

\usepackage{etoolbox}
\patchcmd{\thebibliography}{\section*{\refname}}{}{}{}

\begin{document}
\maketitle

\begin{enumerate}
\item What is the problem that you will be investigating? Why is it interesting?

We are interested in adaptation of the network dissection method presented ~\cite{netdissect2017} in and expanded to GANs in ~\cite{bau2019gandissect} to explore other generative models. When we say other generative models, we most likely mean just another variation of GAN than the progressive GAN they use in the paper, but if the method appears to be adaptable to other models like VAEs (the paper mentions this as a possibility) or transformers ~\cite{DBLP:journals/corr/abs-1904-10509} this could provide interesting results. Ultimately we see the understanding of generative networks as an interesting pursuit and one that will at the very least provide us with a better understanding ourselves.

\item What data will you use?

We will likely use whatever dataset is used by the paper for the model we choose to explore. For example, if we choose to explore the Self-Attention GAN from ~\cite{pmlr-v97-zhang19d}, we'll likely stick with ImageNet that they use. It may also work better to stick with the LSUN datasets used in the GAN Dissection paper ~\cite{bau2019gandissect}. The method requires semantic segmentation of the data, so a dataset like COCO-stuff or ADE20K used by ~\cite{DBLP:journals/corr/abs-1903-07291} (the GauGAN) that comes with semantic classes might prove useful.

\item What method or algorithm are you proposing/modifying from?

We want to adapt the Network Dissection ~\cite{netdissect2017} and GAN Dissection ~\cite{bau2019gandissect} method

\item What reading will you examine to provide context and background?
\begingroup
\renewcommand{\section}[2]{}%
\bibliography{./papers}
\endgroup

\item How will you evaluate your results?

We will observe any noticeable differences in the class representations from those of the proGAN used in ~\cite{bau2019gandissect}. We'll also evaluate our results based on what we learn, since that's most of what we expect to come of this. It could also turn into a replication / reproducability experiment of a paper.

While we wish we could at this point specify a model/paper which we would like to apply the dissection method to, more research is required to determine what type of model is suitable and feasible for this, so that's where we'll start.
\end{enumerate}
\end{document}