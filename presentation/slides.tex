\documentclass[12pt]{beamer}
% Packages
% --------
%	 Necessary
\usepackage{geometry} 						% geometry - page dimensions
\usepackage[parfill]{parskip}				% parskip - to use blank line to sep paragraphs
\usepackage{titling}						% title formatting
\usepackage{enumitem}

\usepackage{amsmath}						% AMS - math, fonts, symbols, theorem
\usepackage{amsfonts}
\usepackage{amssymb}
\usepackage{amsthm}
\usepackage{mathtools}						% mathtools - \coloneqq

\usepackage{tikz}							% tikz - graphs

\usepackage{listings}

\usepackage{hyperref}

% 	Additional
\usepackage{graphicx}						% graphicx - importing graphics from file
\usepackage[T1]{fontenc}
\usepackage[utf8]{inputenc}
%\usepackage{tgbonum}

\usepackage{forest}							% forest - easy trees
\usepackage{tikzsymbols}					% tikzsymbols - just adds some symbols
% tikzlibrary summary: 
% tex.stackexchange.com/questions/42611/list-of-available-tikz-libraries-with-a-short-introduction/491626
\usetikzlibrary{arrows.meta}				% arrows.meta - customizable arrow tips
\usetikzlibrary{positioning}
\usetikzlibrary{shapes}
\usetikzlibrary{shadows}
\usetikzlibrary{automata}
\usetikzlibrary{calc}
\usetikzlibrary{fit}

\usepackage{syntax}

\pagenumbering{gobble}

\tikzset{>={Latex[width=1.5mm, length=2mm]},%
	every state/.style={thick, fill=gray!10},
	initial text=$ $,%
	}
\usepackage{xcolor}							% xcolor - adds additional colors

\usepackage{marginnote}						% marginnote - see name

\usepackage{multirow}						% multirow - sort of a tabular environment of text
\usepackage{bigdelim}						% bigdelim - used with multirow once to make brackets on tables
\usepackage{array}							% array - extends arrary and tabular environment
\usepackage{makecell}						% makecell - adjust cell cizes within tabular environment

\usepackage[bottom]{footmisc}
% Customizations

\pretitle{\begin{center}\Large\bfseries}		% titling format
\posttitle{\par\end{center}\vskip 0cm}
\preauthor{\begin{center}\large}
\postauthor{\end{center}}
\predate{\par\normalsize\centering}
\postdate{\par}

\geometry{margin=1in}

\setlength{\droptitle}{-5em}

\newcommand{\xoverbrace}[2][\vphantom{\dfrac{A}{A}}]{\overbrace{#1#2}}
\newcommand{\xunderbrace}[2][\vphantom{\dfrac{A}{A}}]{\underbrace{#1#2}}

\newtheoremstyle{customnumber} % name
	{}% space above
	{}% space below
	{\normalfont}% body font
	{}% indent 
	{\bfseries}% head font
	{:}% punctuation between head/body
	{ }% space after head: " " = normal whitespace
	{\thmname{#1}\thmnote{ #3}}% head format

\theoremstyle{customnumber}
\newtheorem*{exercise}{Exercise}
\newenvironment{absolutelynopagebreak}
	{\par\nobreak\vfil\penalty0\vfilneg\vtop\bgroup}
	{\par\xdef\tpd{\the\prevdepth}\egroup\prevdepth=\tpd}
	
\theoremstyle{definition}
\newtheorem{defi}{Definition}

\newtheorem{theorem}{Theorem}

\newtheorem{axiom}{Axiom}

\newtheorem{ex}{Example}[section]

\newtheorem{prop}{Proposition}[section]

\newtheoremstyle{named}{}{}{\itshape}{}{\bfseries}{.}{ }{\thmnote{#3}}
\theoremstyle{named}
\newtheorem*{namedtheorem}{}

\newcommand{\stcomp}[1]{{#1}^\complement}

\newcommand{\norm}[1]{\left\lVert#1\right\rVert}

\newcommand{\var}{\mathrm{Var}}
\newcommand{\cov}{\mathrm{Cov}}

\lstset{tabsize=3, numbers=left, basicstyle=\ttfamily, escapeinside=~~, xleftmargin=1cm}
\let\origthelstnumber\thelstnumber
\makeatletter
\newcommand*\Suppressnumber{%
	\lst@AddToHook{OnNewLine}{%
		\let\thelstnumber\relax%
		\advance\c@lstnumber-\@ne\relax%
	}%
}
\newcommand*\Reactivatenumber{%
	\lst@AddToHook{OnNewLine}{%
		\let\thelstnumber\origthelstnumber%
		\advance\c@lstnumber\@ne\relax%
	}%
}
\makeatother

\def\smallqed{\hfill\smash{\scalebox{.5}{$\square$}}}


\DeclareMathOperator{\adj}{adj}
\title{PH}
\author{PH}
\date{}

\begin{document}
\maketitle

\begin{frame}
"Observations of hidden units in large deep neural networks have revealed that human-interpretable concepts sometimes emerge as individual latent variables within those networks"

include introductory network dissection talk because its precursor to GAN dissection but not too much since we covered it in class
\end{frame}

\begin{frame}
Three step process:
\begin{enumerate}
\item Identify a broad set of concepts (segmentation maps), could be specific objects, textures, colors, etc
\item Gather hidden variables' response to known concepts
\item Quantify alignment of hidden variable-concept pairs
\end{enumerate}

"In a fully interpretable local coding such as a one-hot encoding, each variable will match with exactly one concept"

but partially nonlocal representations learned in interior layers, and emergent concepts often align with a combination of several hidden units

\end{frame}

\begin{frame}
Perhaps recap the process of gathering activation maps of each unit, determining quantile $P(\alpha_k > T_k) = 0.005$, upscaling low-resolution activation map to input-resolution annotation mask for a concept, thresholding the upscaled activation map by $T_k$, then evaluating against every concept $c$ in the data set, and then intersection over union score (which has more intuitive understanding visually than formulaic, but it is metrically similar to mutual information)
\end{frame}

Not for presentation, but for project report 
"The IoU evaluating the quality of the segmentation of a unit is an objective confidence score for interpretability that is comparable across networks."


\end{document}