\documentclass{article}

%% Packages
% --------
%	 Necessary
\usepackage{geometry} 						% geometry - page dimensions
\usepackage[parfill]{parskip}				% parskip - to use blank line to sep paragraphs
\usepackage{titling}						% title formatting
\usepackage{enumitem}

\usepackage{amsmath}						% AMS - math, fonts, symbols, theorem
\usepackage{amsfonts}
\usepackage{amssymb}
\usepackage{amsthm}
\usepackage{mathtools}						% mathtools - \coloneqq

\usepackage{tikz}							% tikz - graphs

\usepackage{listings}

\usepackage{hyperref}

% 	Additional
\usepackage{graphicx}						% graphicx - importing graphics from file
\usepackage[T1]{fontenc}
\usepackage[utf8]{inputenc}
%\usepackage{tgbonum}

\usepackage{forest}							% forest - easy trees
\usepackage{tikzsymbols}					% tikzsymbols - just adds some symbols
% tikzlibrary summary: 
% tex.stackexchange.com/questions/42611/list-of-available-tikz-libraries-with-a-short-introduction/491626
\usetikzlibrary{arrows.meta}				% arrows.meta - customizable arrow tips
\usetikzlibrary{positioning}
\usetikzlibrary{shapes}
\usetikzlibrary{shadows}
\usetikzlibrary{automata}
\usetikzlibrary{calc}
\usetikzlibrary{fit}

\usepackage{syntax}

\pagenumbering{gobble}

\tikzset{>={Latex[width=1.5mm, length=2mm]},%
	every state/.style={thick, fill=gray!10},
	initial text=$ $,%
	}
\usepackage{xcolor}							% xcolor - adds additional colors

\usepackage{marginnote}						% marginnote - see name

\usepackage{multirow}						% multirow - sort of a tabular environment of text
\usepackage{bigdelim}						% bigdelim - used with multirow once to make brackets on tables
\usepackage{array}							% array - extends arrary and tabular environment
\usepackage{makecell}						% makecell - adjust cell cizes within tabular environment

\usepackage[bottom]{footmisc}
% Customizations

\pretitle{\begin{center}\Large\bfseries}		% titling format
\posttitle{\par\end{center}\vskip 0cm}
\preauthor{\begin{center}\large}
\postauthor{\end{center}}
\predate{\par\normalsize\centering}
\postdate{\par}

\geometry{margin=1in}

\setlength{\droptitle}{-5em}

\newcommand{\xoverbrace}[2][\vphantom{\dfrac{A}{A}}]{\overbrace{#1#2}}
\newcommand{\xunderbrace}[2][\vphantom{\dfrac{A}{A}}]{\underbrace{#1#2}}

\newtheoremstyle{customnumber} % name
	{}% space above
	{}% space below
	{\normalfont}% body font
	{}% indent 
	{\bfseries}% head font
	{:}% punctuation between head/body
	{ }% space after head: " " = normal whitespace
	{\thmname{#1}\thmnote{ #3}}% head format

\theoremstyle{customnumber}
\newtheorem*{exercise}{Exercise}
\newenvironment{absolutelynopagebreak}
	{\par\nobreak\vfil\penalty0\vfilneg\vtop\bgroup}
	{\par\xdef\tpd{\the\prevdepth}\egroup\prevdepth=\tpd}
	
\theoremstyle{definition}
\newtheorem{defi}{Definition}

\newtheorem{theorem}{Theorem}

\newtheorem{axiom}{Axiom}

\newtheorem{ex}{Example}[section]

\newtheorem{prop}{Proposition}[section]

\newtheoremstyle{named}{}{}{\itshape}{}{\bfseries}{.}{ }{\thmnote{#3}}
\theoremstyle{named}
\newtheorem*{namedtheorem}{}

\newcommand{\stcomp}[1]{{#1}^\complement}

\newcommand{\norm}[1]{\left\lVert#1\right\rVert}

\newcommand{\var}{\mathrm{Var}}
\newcommand{\cov}{\mathrm{Cov}}

\lstset{tabsize=3, numbers=left, basicstyle=\ttfamily, escapeinside=~~, xleftmargin=1cm}
\let\origthelstnumber\thelstnumber
\makeatletter
\newcommand*\Suppressnumber{%
	\lst@AddToHook{OnNewLine}{%
		\let\thelstnumber\relax%
		\advance\c@lstnumber-\@ne\relax%
	}%
}
\newcommand*\Reactivatenumber{%
	\lst@AddToHook{OnNewLine}{%
		\let\thelstnumber\origthelstnumber%
		\advance\c@lstnumber\@ne\relax%
	}%
}
\makeatother

\def\smallqed{\hfill\smash{\scalebox{.5}{$\square$}}}


\DeclareMathOperator{\adj}{adj}

% if you need to pass options to natbib, use, e.g.:
%     \PassOptionsToPackage{numbers, compress}{natbib}
% before loading neurips_2019

% ready for submission
% \usepackage{neurips_2019}

% to compile a preprint version, e.g., for submission to arXiv, add add the
% [preprint] option:
     \usepackage[preprint]{neurips_2019}

% to compile a camera-ready version, add the [final] option, e.g.:
%    \usepackage[final]{neurips_2019}

% to avoid loading the natbib package, add option nonatbib:
%     \usepackage[nonatbib]{neurips_2019}

\usepackage[utf8]{inputenc} % allow utf-8 input
\usepackage[T1]{fontenc}    % use 8-bit T1 fonts
\usepackage{hyperref}       % hyperlinks
\usepackage{url}            % simple URL typesetting
\usepackage{booktabs}       % professional-quality tables
\usepackage{amsfonts}       % blackboard math symbols
\usepackage{nicefrac}       % compact symbols for 1/2, etc.
\usepackage{microtype}      % microtypography

\bibliographystyle{unsrtnat}

\title{GAN Dissecting aGAN}

% The \author macro works with any number of authors. There are two commands
% used to separate the names and addresses of multiple authors: \And and \AND.
%
% Using \And between authors leaves it to LaTeX to determine where to break the
% lines. Using \AND forces a line break at that point. So, if LaTeX puts 3 of 4
% authors names on the first line, and the last on the second line, try using
% \AND instead of \And before the third author name.

\author{%
  Tripp Isbell\\
  Auburn University\\
  \texttt{cai0004@auburn.edu} \\
  %% Add other stamps here
  \And
  Placeholder \\
  \And
  Placeholder \\
  % examples of more authors
  % \And
  % Coauthor \\
  % Affiliation \\
  % Address \\
  % \texttt{email} \\
  % \AND
  % Coauthor \\
  % Affiliation \\
  % Address \\
  % \texttt{email} \\
  % \And
  % Coauthor \\
  % Affiliation \\
  % Address \\
  % \texttt{email} \\
  % \And
  % Coauthor \\
  % Affiliation \\
  % Address \\
  % \texttt{email} \\
}

\begin{document}

\maketitle

\begin{abstract}
We explore the GAN Dissection method and stuff.
\end{abstract}

\section{Introduction}

%% placeholder text below

Much research in the field of Deep Learning has worked towards the goal of visualizing and understanding deep neural network architectures (cite much research). Introduce Network Dissection as a means of understanding semantic representation of Convolutional Neural Networks (cite net dissect). 

The field of generative models is a-boomin', and from it many unique models have emerged, including (cite unique models). A dominant/popular model for image generation is the Generative Adversarial Network (cite GANs) and many GAN variants (cite GAN variants, or maybe don't, probably don't blow all our citations so early). All of this research/progress has spawned new research into understanding generative models. And maybe go into some research on generative models (with citations). Since the generative models use 

cite spam to be copied:
\citet{netdissect2017}
\cite{netdissect2017}
\citet{gandissect2019}
\cite{gandissect2019}
\citet{progan2017}
\cite{progan2017}
\citet{stylegan2018}
\cite{stylegan2018}
\citet{gan2014}
\cite{gan2014}
\citet{synthesizing2016}
\cite{synthesizing2016}
\citet{deepvis2015}
\cite{deepvis2015}

\section{GAN Dissection}

GAN Dissection is accomplished in two parts: dissection and intervention. 

\subsection{Applications}

%% TODO

Talk about our hypotheses about their method, how we seek to apply it to study other GAN architectures and see if anything interesting happens, etc.


\section{StyleGAN}

Discuss the StyleGAN paper, architecture, their results and experiments, and why we think it would be a good candidate architecture for Dissection and what we might learn

\section{Methods}

Maybe discuss the GAN Dissection methods in brief

\subsection{Progressive GANs}

We first replicate (one of) the GAN Dissection experiments of the paper. Discuss details such as datasets used, ProGAN details, dissection details (layer, whatever other adjustable parameters the tool has), oh yeah mention that we use the tool provided at \url{http://github.com/CSAILVision/GANDissect} to do the heavy lifting. 

\subsection{StyleGAN }

Same stuff

\section{Results}

More to come.

\section{Analysis}

More to come.

\section{Discussion}

More to come.

\subsubsection*{Acknowledgments}

\bibliography{references}

\end{document}